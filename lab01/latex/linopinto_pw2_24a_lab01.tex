%package list
\documentclass{article}
\usepackage[top=3cm, bottom=3cm, outer=3cm, inner=3cm]{geometry}
\usepackage{multicol}
\usepackage{graphicx}
\usepackage{url}
%\usepackage{cite}
\usepackage{hyperref}
\usepackage{array}
%\usepackage{multicol}
\newcolumntype{x}[1]{>{\centering\arraybackslash\hspace{0pt}}p{#1}}
\usepackage{natbib}
\usepackage{pdfpages}
\usepackage{multirow}
\usepackage[normalem]{ulem}
\useunder{\uline}{\ul}{}
\usepackage{svg}
\usepackage{xcolor}
\usepackage{listings}
\lstdefinestyle{ascii-tree}{
    literate={├}{|}1 {─}{--}1 {└}{+}1 
  }
\lstset{basicstyle=\ttfamily,
  showstringspaces=false,
  commentstyle=\color{red},
  keywordstyle=\color{blue}
}
%\usepackage{booktabs}
\usepackage[labelformat=empty]{caption}
\usepackage{subcaption}
\usepackage{float}
\usepackage{array}

\newcolumntype{M}[1]{>{\centering\arraybackslash}m{#1}}
\newcolumntype{N}{@{}m{0pt}@{}}


%%%%%%%%%%%%%%%%%%%%%%%%%%%%%%%%%%%%%%%%%%%%%%%%%%%%%%%%%%%%%%%%%%%%%%%%%%%%
%%%%%%%%%%%%%%%%%%%%%%%%%%%%%%%%%%%%%%%%%%%%%%%%%%%%%%%%%%%%%%%%%%%%%%%%%%%%
\newcommand{\itemEmail}{mjarama@unsa.edu.pe}
\newcommand{\itemStudent}{Mariel Alisson Jara Mamani}
\newcommand{\itemCourse}{Programación Web 2}
\newcommand{\itemCourseCode}{1702122}
\newcommand{\itemSemester}{I}
\newcommand{\itemUniversity}{Universidad Nacional de San Agustín de Arequipa}
\newcommand{\itemFaculty}{Facultad de Ingeniería de Producción y Servicios}
\newcommand{\itemDepartment}{Departamento Académico de Ingeniería de Sistemas e Informática}
\newcommand{\itemSchool}{Escuela Profesional de Ingeniería de Sistemas}
\newcommand{\itemAcademic}{2023 - B}
\newcommand{\itemInput}{Del 30 Abril 2024}
\newcommand{\itemOutput}{Al 04 Mayo 2024}
\newcommand{\itemPracticeNumber}{01}
\newcommand{\itemTheme}{Docker}
%%%%%%%%%%%%%%%%%%%%%%%%%%%%%%%%%%%%%%%%%%%%%%%%%%%%%%%%%%%%%%%%%%%%%%%%%%%%
%%%%%%%%%%%%%%%%%%%%%%%%%%%%%%%%%%%%%%%%%%%%%%%%%%%%%%%%%%%%%%%%%%%%%%%%%%%%

\usepackage[english,spanish]{babel}
\usepackage[utf8]{inputenc}
\AtBeginDocument{\selectlanguage{spanish}}
\renewcommand{\figurename}{Figura}
\renewcommand{\refname}{Referencias}
\renewcommand{\tablename}{Tabla} %esto no funciona cuando se usa babel
\AtBeginDocument{%
	\renewcommand\tablename{Tabla}
}

\usepackage{fancyhdr}
\pagestyle{fancy}
\fancyhf{}
\setlength{\headheight}{30pt}
\renewcommand{\headrulewidth}{1pt}
\renewcommand{\footrulewidth}{1pt}
\fancyhead[L]{\raisebox{-0.2\height}{\includegraphics[width=3cm]{img/logo_episunsa.png}}}
\fancyhead[C]{\fontsize{7}{7}\selectfont	\itemUniversity \\ \itemFaculty \\ \itemDepartment \\ \itemSchool \\ \textbf{\itemCourse}}
\fancyhead[R]{\raisebox{-0.1\height}{\includegraphics[width=1.2cm]{img/logo_abet}}}
\fancyfoot[L]{Mariel Jara}
\fancyfoot[C]{\itemCourse}
\fancyfoot[R]{Página \thepage}

% para el codigo fuente
\usepackage{listings}
\usepackage{color, colortbl}
\definecolor{dkgreen}{rgb}{0,0.6,0}
\definecolor{gray}{rgb}{0.5,0.5,0.5}
\definecolor{mauve}{rgb}{0.58,0,0.82}
\definecolor{codebackground}{rgb}{0.95, 0.95, 0.92}
\definecolor{tablebackground}{rgb}{0.8, 0, 0}

\lstset{frame=tb,
	language=bash,
	aboveskip=3mm,
	belowskip=3mm,
	showstringspaces=false,
	columns=flexible,
	basicstyle={\small\ttfamily},
	numbers=none,
	numberstyle=\tiny\color{gray},
	keywordstyle=\color{blue},
	commentstyle=\color{dkgreen},
	stringstyle=\color{mauve},
	breaklines=true,
	breakatwhitespace=true,
	tabsize=3,
	backgroundcolor= \color{codebackground},
}

\begin{document}

\vspace*{10px}

\begin{center}
	\fontsize{17}{17} \textbf{ Informe de Laboratorio \itemPracticeNumber}
\end{center}
\centerline{\textbf{\Large Tema: \itemTheme}}
%\vspace*{0.5cm}	

\begin{flushright}
	\begin{tabular}{|M{2.5cm}|N|}
		\hline
		\rowcolor{tablebackground}
		\color{white} \textbf{Nota} \\
		\hline
		\\[30pt]
		\hline
	\end{tabular}
\end{flushright}

\begin{table}[H]
	\begin{tabular}{|M{4.7cm}|M{4.7cm}|M{4.7cm}|}
		\hline
		\rowcolor{tablebackground}
		\color{white} \textbf{Estudiante} & \color{white}\textbf{Escuela} & \color{white}\textbf{Asignatura}                                        \\
		\hline
		{\itemStudent \par \itemEmail}    & \itemSchool                   & {\itemCourse \par Semestre: \itemSemester \par Código: \itemCourseCode} \\
		\hline
	\end{tabular}
\end{table}

\begin{table}[H]
	\begin{tabular}{|M{4.7cm}|M{4.7cm}|M{4.7cm}|}
		\hline
		\rowcolor{tablebackground}
		\color{white}\textbf{Laboratorio} & \color{white}\textbf{Tema} & \color{white}\textbf{Duración} \\
		\hline
		\itemPracticeNumber               & \itemTheme                 & 04 horas                       \\
		\hline
	\end{tabular}
\end{table}

\begin{table}[H]
	\begin{tabular}{|M{4.7cm}|M{4.7cm}|M{4.7cm}|}
		\hline
		\rowcolor{tablebackground}
		\color{white}\textbf{Semestre académico} & \color{white}\textbf{Fecha de inicio} & \color{white}\textbf{Fecha de entrega} \\
		\hline
		\itemAcademic                            & \itemInput                            & \itemOutput                            \\
		\hline
	\end{tabular}
\end{table}

%%%%%%%%%%%%%%%%%%%%%%%%%%%%%%%%%%%%%%%%%%%%%%%%%%%%%%%%%%%%%%%%%%%%%%
\section{Tarea}
\begin{itemize}
	\item \textbf{Procedimiento:}
	      \\- Instale el servidor web Apache HTTP server 2.x
		  \\- Instale cualquiera de estos lenguajes de programación: PHP, Perl, Python.
		  \\- Configure el servidor web para que interprete uno de los lenguajes de programación.
		  \\- Instale cualquiera de los servidores de base de datos: MySQL, MariaDB, PostgreSQL.
		  \\- Instale el servidor Open SSH Server. Envíe archivos al servidor: imágenes, css, js, etc.
		  \\- Cree un usuario pw2 con contraseña: 12345678.
		  \\- Otorgue permisos al usuario para acceder a la aplicación web. (Read/Write)
		  \\- Finalmente implemente el trabajo final del curso de pw1 en ese contenedor.
		  \\- Elabore un informe paso a paso para donde explique funcionalmente el proyecto demostrando que se trata de un contenedor docker.
		  \\- Adjunte la URL de un video donde muestre que se trata de un contenedor Docker.
	\item \textbf{Entregables:}
	      \\- Informe de laboratorio.
		  \\- URL: Video Youtube.
		  \\- INVESTIGACIÓN: Suba el proyecto a Docker Hub (Público).
		  \\- URL: Docker Hub. 
\end{itemize}
%%%%%%%%%%%%%%%%%%%%%%%%%%%%%%%%%%%%%%%%%%%%%%%%%%%%%%%%%%%%%%%%%%%%%%
\pagebreak

\section{Equipos y materiales utilizados}
\begin{itemize}
	%%%%%%%%%%%%%%%%%%%%%%%%%%%%%%%%%%%%%%%%%%%%%%%%%%%%%%%%%%%%%%%%%%%%%%
	\item Sistema Operativo Microsoft Windows 10
	\item Visual Studio Code
	%\item Git 2.41.0.windows.1
	\item Windows PowerShell
	%\item Cuenta en GitHub con el correo institucional.
	\item Docker con imagen Ubuntu
	\item Servidor web Apache HTTP
	\item Lenguaje de programación Perl
	\item Servidor de base de datos MariaDB
	\item Servidor Open SSH Server
	\item Proyecto final del curso de Programación Web 1 
	\item WinSCP
	\item Vim
	\item OBS
	%%%%%%%%%%%%%%%%%%%%%%%%%%%%%%%%%%%%%%%%%%%%%%%%%%%%%%%%%%%%%%%%%%%%%%
\end{itemize}

\section{Procedimiento}

\section{URL del video}
\begin{itemize}
	\item \url{https://www.youtube.com/watch?v=1urjqNs_kT8}
\end{itemize}

\section{URL de Docker Hub}
\begin{itemize}
	\item \url{https://hub.docker.com/r/marielj/lab01/tags}
\end{itemize}

\section{Estructura de laboratorio \itemPracticeNumber}
\begin{itemize}
	\item El contenido que se entrega en este laboratorio es el siguiente:
\end{itemize}
%%%%%%%%%%%%%%%%%%%%%%%%%%%%%%%%%%%%%%%%%%%%%%%%%%%%%%%%%%%%%%%%%%%%%%
\begin{lstlisting}[style=ascii-tree]
lab01/
|--- README.md
|--- latex/
	   |--- linopinto_pw2_24a_lab01.tex
	   |--- linopinto_pw2_24a_lab01.pdf
          |--- img/
	       |--- logo_abet.png
	       |--- logo_episunsa.png
\end{lstlisting}
%%%%%%%%%%%%%%%%%%%%%%%%%%%%%%%%%%%%%%%%%%%%%%%%%%%%%%%%%%%%%%%%%%%%%%

\section{Referencias}
\begin{itemize}
	\item \url{https://docs.docker.com/}
\end{itemize}

%\pagebreak
%\bibliographystyle{apalike}
%\bibliographystyle{IEEEtranN}
%\bibliography{bibliography}

\end{document}